\documentclass[12pt]{amsart}
\usepackage{amsmath}
\usepackage[margin=0.5in]{geometry}
\usepackage{graphicx}
\graphicspath{{.}}
\begin{document}
\centerline {\bf Selected solutions to Kenneth H. Rosen's}
\centerline{\bf Elementary Number Theory and Its Applications (5th edition).}
\centerline {\small Larkin Wisdom}
\vskip 1 cm
\noindent{\bf 1.1.5.} Use the well-ordering property to show that $\sqrt{3}$ is irrational.
\vskip .5 cm
{\bf Solution.} Assume to the contrary that $\sqrt{3}$ is rational. Then $\sqrt{3} = \frac{a}{b}$ for some $a, b\in \mathbb{N}$. Consider the set $S=\{k\sqrt{3}\:\:|\:\: k, k\sqrt{3}\in\mathbb{N}\}$. This set is nonempty, as $a = b\sqrt{3}\in S$. As $S$ is a nonempty set of positive integers, by the well-ordering property $S$ has a least element, $s\sqrt{3}$. Now, consider $t = (s\sqrt{3} - s)\sqrt{3}$. As $s\sqrt{3} - s$ is a positive integer, $t\in S$. But $s\sqrt{3} - s < s$, and so $t$ is smaller than the supposed smallest element of $S$ ($s\sqrt{3}$), a contradiction.
\end{document}